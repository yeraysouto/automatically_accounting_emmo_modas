\documentclass[11pt,a4paper]{article}

\usepackage[T1]{fontenc}
\usepackage[utf8]{inputenc}
\usepackage[spanish]{babel}
\usepackage{lmodern}
\usepackage{geometry}
\usepackage{hyperref}
\usepackage{booktabs}
\usepackage{longtable}
\usepackage{enumitem}
\usepackage{microtype}
\usepackage{listings}
\usepackage{xcolor}

\geometry{margin=2.2cm}
\hypersetup{colorlinks=true, linkcolor=blue, urlcolor=blue, citecolor=blue}

\definecolor{codebg}{RGB}{248,248,248}
\definecolor{codegray}{RGB}{90,90,90}

\lstset{
  basicstyle=\ttfamily\small,
  backgroundcolor=\color{codebg},
  frame=single,
  framerule=0.2pt,
  rulecolor=\color{codegray},
  breaklines=true,
  columns=fullflexible,
  keepspaces=true,
  showstringspaces=false
}

\title{Respuesta T\'ecnica (Auditor\'ia TI): Seguridad, JSON, Almacenamiento, Pricing y Flujo de Datos}
\author{Proyecto EMMA / EMMO}
\date{\today}

\begin{document}
\maketitle

\section{Contexto}
Este documento responde, con terminolog\'ia de ingenier\'ia y auditor\'ia TI, a los comentarios sobre:
\begin{itemize}
  \item estructura del proyecto, ciberseguridad y gesti\'on de \texttt{.env};
  \item organizaci\'on del JSON para OCR (multi-tipo de factura); 
  \item gesti\'on y retenci\'on de PDFs por trimestre;
  \item logging, observabilidad y pol\'itica de fallos;
  \item pricing con reglas matem\'aticas sin IA;
  \item migrabilidad SQLite $\rightarrow$ Postgres;
  \item nomenclatura del modelo de datos y trazabilidad;
  \item c\'omo pasa \texttt{ocr\_info\_clothes} a ``Importaci\'on Art\'iculos Montcau''.
\end{itemize}

\section{2. Estructura del proyecto, ciberseguridad y \texttt{.env}}
\subsection{Estructura y responsabilidades}
El repositorio separa responsabilidades:
\begin{itemize}
  \item \texttt{backend/}: API de negocio (FastAPI), modelos SQLAlchemy, servicios (OCR, storage, pricing), tests.
  \item \texttt{docs/}: documentaci\'on t\'ecnica (\LaTeX) y material de requisitos.
  \item \texttt{automate\_reading\_accountant.js}: prototipo/legado para pruebas iniciales; \textbf{no} se ejecuta en producci\'on ni es dependencia del backend.
\end{itemize}

\subsection{Gesti\'on de secretos y configuraci\'on (\texttt{.env})}
Se utiliza \texttt{.env} \textbf{s\'olo} para desarrollo local. En producci\'on la configuraci\'on debe provenir del entorno (CI/CD, contenedor, secret manager).

Variables relevantes (prefijo \texttt{EMMO\_}):
\begin{itemize}
  \item \texttt{EMMO\_DATABASE\_URL}: string de conexi\'on (SQLite por defecto; Postgres en prod).
  \item \texttt{EMMO\_API\_KEY}: secreto est\'atico para autenticar clientes.
  \item \texttt{EMMO\_REQUIRE\_API\_KEY\_FOR\_WRITE}: exige API key para escritura (POST/PUT).
  \item \texttt{EMMO\_REQUIRE\_API\_KEY\_FOR\_READ}: (opcional) exige API key para lectura (GET) cuando la API se expone.
\end{itemize}

\subsection{Ciberseguridad: c\'omo se protegen las APIs (MVP + ruta a prod)}
\textbf{MVP (ya implementado):}
\begin{itemize}
  \item \textbf{Autenticaci\'on por API key} en cabecera \texttt{X-API-Key}.
  \item \textbf{Comparaci\'on en tiempo constante} (mitiga timing attacks en verificaci\'on de secretos).
  \item \textbf{Security headers} (hardening del navegador): \texttt{X-Content-Type-Options}, \texttt{X-Frame-Options}, \texttt{Referrer-Policy}, \texttt{Permissions-Policy}.
  \item \textbf{Request correlation}: se propaga \texttt{X-Request-ID} y se inyecta en logs para trazabilidad.
  \item \textbf{Rate limiting b\'asico} (in-memory) configurable por minuto; \textbf{no} multi-instancia.
  \item \textbf{Trusted hosts} (opcional) para mitigar host-header attacks si se expone directamente.
\end{itemize}

\textbf{Producci\'on (recomendaci\'on de auditor\'ia):}
\begin{itemize}
  \item Terminar TLS (HTTPS) en reverse proxy/API gateway.
  \item WAF / rate limit centralizado (Redis/gateway) si se escala horizontalmente.
  \item Secret manager (no \texttt{.env} en disco) y rotaci\'on del API key.
  \item Segmentaci\'on de red (allowlist) y CORS estricto.
\end{itemize}

\section{JSON: organizaci\'on, multi-tipo de factura y consistencia}
\subsection{Dise\~no recomendado}
Para no ``confundir modelos'' ni forzar claves vac\'ias:
\begin{enumerate}
  \item Separar \textbf{cabecera} y \textbf{l\'ineas}.
  \item Declarar \textbf{tipo de factura} en \texttt{invoice.invoice\_type}.
  \item Guardar campos variables de negocio en \texttt{invoice.optional\_fields} (diccionario libre).
\end{enumerate}

\subsection{Contrato de OCR (payload normalizado)}
Ejemplo de shape can\'onico:
\begin{lstlisting}
{
  "invoice": {
    "cif_supplier": "B123...",
    "name_supplier": "Proveedor SL",
    "num_invoice": "F-2026-001",
    "date": "2026-01-16",
    "total_invoice_amount": 123.45,
    "invoice_type": "textil",
    "optional_fields": {"campana": "rebajas"},
    "raw_text": "..."
  },
  "lines": [
    {
      "reference_code": "ABC123",
      "description": "CAMISETA",
      "quantity": 10,
      "price": 5.5,
      "total_no_iva": 55.0
    }
  ]
}
\end{lstlisting}

\subsection{Referencia: evitar colisiones y claves vac\'ias}
Para manejar facturas con referencias incompletas y evitar colisiones:
\begin{itemize}
  \item \texttt{reference\_code\_raw}: lo que vino del OCR/manual.
  \item \texttt{reference\_code}: formato can\'onico \texttt{SUP\_\_\_\_\_} + c\'odigo (prefijo 3 letras del proveedor).
  \item Si no hay referencia: se habilita un fallback determinista \texttt{SUP\_\textless hash64\textgreater} o \textbf{human-in-the-loop} para completarla.
\end{itemize}

\section{Invoices: gesti\'on y retenci\'on de PDFs por trimestre}
\subsection{Decisi\'on de auditor\'ia}
\textbf{S\'i, se guardan} los PDFs/imagenes para trazabilidad, auditor\'ia y presentaci\'on trimestral.

\subsection{Estrategia de almacenamiento}
Al subir una factura, el backend persiste el fichero en:\\
\texttt{<storage\_root>/invoices/<year>/Q<quarter>/<uuid>.<ext>}

Metadatos en BBDD:
\begin{itemize}
  \item ruta relativa, nombre, MIME;
  \item \textbf{sha256} del contenido (integridad / no repudio interno);
  \item tama\~no en bytes.
\end{itemize}

\section{Tests: logging, seguimiento y pol\'itica de fallos}
\subsection{Logs y observabilidad}
Los logs se emiten a \textbf{stdout} (patr\'on cloud-native) y se pueden serializar como JSON.

Seguimiento recomendado:
\begin{itemize}
  \item Agregador (ELK/Datadog/Grafana Loki) capturando stdout.
  \item Correlaci\'on por \texttt{X-Request-ID}.
  \item Alarmas por ratio de 4xx/5xx y por \texttt{last\_error\_code}.
\end{itemize}

\subsection{Fallo blando vs fallo duro (no parar el negocio)}
\textbf{Fallo duro (bloquea request / para el flujo):}
\begin{itemize}
  \item autenticaci\'on (401),
  \item MIME no permitido (415),
  \item exceder tama\~no (413),
  \item conflictos de integridad (409) donde el dato ser\'ia inconsistente.
\end{itemize}

\textbf{Fallo blando (no para sistema; se registra para corregir):}
\begin{itemize}
  \item fallo del OCR externo: se crea factura con \texttt{status=needs\_review} y se rellenan \texttt{last\_error\_*}.
  \item l\'ineas sin \texttt{reference\_code}: se ingieren y se corrigen v\'ia human-in-the-loop.
\end{itemize}

\section{Pricing: correcci\'on de precios demasiado bajos (sin IA)}
\subsection{Criterios y fuente de verdad}
Dado que las descripciones son ruidosas y no hay SKU, el anclaje debe ser \textbf{ref\_code}.

Criterios (reglas matem\'aticas):
\begin{enumerate}
  \item Si existe maestro de art\'iculos con \texttt{coste\_unitario}: validar que \texttt{price} $\ge$ \texttt{coste\_unitario} $\times$ \texttt{min\_ratio}.
  \item Si no existe coste maestro: usar un hist\'orico por \texttt{reference\_code} (mediana) y validar contra \texttt{median} $\times$ \texttt{min\_ratio}.
\end{enumerate}

\subsection{\guillemotleft Necesitas una BBDD de antiguos precios?\guillemotright}
S\'i: para el segundo caso. Se mantiene una tabla de observaciones (precio, referencia, timestamp). Esto permite reglas por negocio (p.ej. ventanas trimestrales, percentiles, etc.) sin IA.

\section{Arquitectura de BBDD: SQLite hoy, Postgres ma\~nana}
Se usa SQLite para iterar r\'apido; el acceso est\'a abstra\'ido por SQLAlchemy y se parametriza por \texttt{EMMO\_DATABASE\_URL}. En producci\'on se recomienda Postgres por concurrencia, locking y operaci\'on.

\section{5. Modelo de datos (nomenclatura, PK/FK, trazabilidad)}
\subsection{5.1 Cabecera (proveedor/factura)}
Decisiones:
\begin{itemize}
  \item \texttt{name\_supplier}: string.
  \item \texttt{tel\_number\_supplier}: \textbf{string} (internacionalizaci\'on, prefijos, formatos).
  \item \texttt{email\_supplier}: string.
  \item \texttt{num\_invoice}: string.
  \item \texttt{total\_invoice\_amount}: evita ambig\"uedad con SQL (m\'as claro que \texttt{total\_supplier}).
  \item Trazabilidad de mensajes: \texttt{source\_channel}, \texttt{source\_thread\_id}, \texttt{source\_message\_id}.
\end{itemize}

\subsection{5.2 PK y FK (glosario)}
\begin{itemize}
  \item \textbf{PK (Primary Key)}: identificador \`unico de una fila (ej. \texttt{data\_ocr\_invoice.id}).
  \item \textbf{FK (Foreign Key)}: referencia a otra tabla (ej. \texttt{ocr\_info\_clothes.invoice\_id} apunta a la PK de factura).
\end{itemize}

\subsection{ref\_code: m\'ultiples estrategias (OCR + hash64 + human-in-the-loop)}
Estrategia gradual:
\begin{enumerate}
  \item OCR intenta extraer \texttt{reference\_code}.
  \item Si viene, se normaliza a \texttt{SUP\_\_\_ + ref\_code} (prefijo 3 letras + ``\_'' + c\'odigo).
  \item Si no viene, se genera \texttt{SUP\_\textless hash64\textgreater} (determinista) o se solicita al operador (human-in-the-loop).
\end{enumerate}

\section{OCR: Tesseract}
Tesseract es una buena opci\'on para un primer escal\'on local (im\'agen) y como fallback. En PDFs complejos suele requerir pipeline extra (render PDF $\rightarrow$ imagen $\rightarrow$ OCR) o un proveedor especializado.

\section{Paso de \texttt{ocr\_info\_clothes} a Importaci\'on Art\'iculos Montcau}
\subsection{Modelo de integraci\'on}
Se implementa una exportaci\'on en JSON (preparado para transformaci\'on a CSV o API externa):
\begin{itemize}
  \item Se seleccionan l\'ineas de la factura.
  \item Se filtran las que tienen \texttt{reference\_code} (las dem\'as quedan para human-in-the-loop).
  \item Se proyectan campos a ``fila Montcau'': \texttt{reference\_code}, \texttt{descripcion}, \texttt{cantidad}, \texttt{coste\_unitario}.
\end{itemize}

\subsection{Endpoint de exportaci\'on (backend)}
\begin{lstlisting}
GET /invoices/{invoice_id}/export/importacion-montcau
\end{lstlisting}

\subsection{Integraci\'on con PrestaShop (siguiente paso)}
Dado que esta semana hay reuniones con el equipo de integraci\'on, el enfoque recomendado es:
\begin{itemize}
  \item acordar el \textbf{contrato} (campos obligatorios, encoding, unidad de medida, reglas de merge),
  \item decidir si la integraci\'on es por:\\
    (a) import CSV, (b) API interna Montcau, (c) API PrestaShop (webservice) + mapeo a productos,
  \item mantener el backend como \textbf{fuente de trazabilidad} (auditable) y la integraci\'on como capa de entrega.
\end{itemize}

\section{Comandos de compilaci\'on (terminal)}
Recomendado (latexmk):
\begin{lstlisting}
cd docs
latexmk -pdf respuesta_auditoria_ti.tex
\end{lstlisting}

Alternativa (sin latexmk):
\begin{lstlisting}
cd docs
pdflatex respuesta_auditoria_ti.tex
pdflatex respuesta_auditoria_ti.tex
\end{lstlisting}

\end{document}
